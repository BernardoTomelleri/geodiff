\documentclass[italian,a4paper,10pt]{article}

% path to figure folder
\graphicspath{{figs/}}

\newtheorem{lem}{Lemma}[section]
\newtheorem{definizione}{Definizione}[section]
\newtheorem{teorema}{Teorema}[section]
\newtheorem{pro}{Proposizione}[section]
\newtheorem{cor}{Corollario}[section]
\declaretheoremstyle[
spaceabove=6pt, spacebelow=6pt,
headfont=\normalfont\bfseries,
notefont=\mdseries, notebraces={(}{)},
bodyfont=\normalfont,
postheadspace=1em,
qed=,
]{exercise_style}
\declaretheorem[style=exercise_style, name=Esempio, numberwithin=section]{example}
\declaretheorem[style=exercise_style, name=Esercizio, 
numberwithin=section]{exercise}
\declaretheorem[style=exercise_style, name=Problema,
numberwithin=subsection]{problem}
\declaretheorem[style=exercise_style, name=Osservazione, 
numberwithin=section]{observation}

% abbreviated labels lowercase c
\crefname{lem}{Lem.}{Lemm.}
\crefname{definizione}{Def.}{Def.ni}
\crefname{teorema}{Teo.}{Teo.mi}
\crefname{pro}{Prop.}{Propp.}
\crefname{cor}{Cor.}{Corr.}
% non-abbreviated labels uppercase C
\Crefname{lem}{Lemma}{Lemmi}
\Crefname{definizione}{Definizione}{Definizioni}
\Crefname{teorema}{Teorema}{Teoremi}
\Crefname{pro}{Proposizione}{Proposizioni}
\Crefname{cor}{Corollario}{Corollari}

% turns all (hyperlinked) references black [default is blue]
\hypersetup{
	linktoc=all,
	colorlinks=true,
	linkcolor=black
}

% imports math shorthand notation
% abbreviation for (sub_/super^)scripts of \lim, \sum,... in inline math
\newcommand{\ds}{\displaystyle}
\newcommand{\numberset}{\mathbb}
\newcommand{\C}{\numberset{C}}
\newcommand{\R}{\numberset{R}}
\newcommand{\Q}{\numberset{Q}}
\newcommand{\Z}{\numberset{Z}}
\newcommand{\N}{\numberset{N}}
\newcommand{\PP}{\mathbb{P}}
\DeclarePairedDelimiter{\norm}{\lVert}{\rVert}
\DeclarePairedDelimiter{\scalar}{\langle}{\rangle}

% Logarithm with arbitrary base.
% -> log_10
\newcommand{\llog}[1][10]{\log_{#1}}

% Absolute value.
% -> |x|
\newcommand{\abs}[1]{\left| #1 \right|}

% Powers.
% -> x^a
\newcommand{\power}[2][2]{\left( #2 \right)^{#1}}

% Square.
% -> x^2
\newcommand{\sq}[1]{\power[2]{#1}}

% Expansion of the binomial coefficient.
% -> n1!/(n2!(n1 - n2)!)
\newcommand{\binomexpr}[2]{\frac{#1!}{#2!(#1 - #2)!}}

% Expression evaluation at a given point with square brackets.
% -> [x]_{a}
\newcommand{\at}[2]{\left[ #1\right]_{\makebox[-1pt][l]{${\scriptstyle#2}$}}}

% Expression evaluation in an interval.
% -> [x] _{a}^{b}
\newcommand{\eval}[3]{\left.#1%
  \right|_{\makebox[-1pt][l]{${\scriptstyle#2}$}}^{\makebox[-1pt][l]{${\scriptstyle#3}$}}}

% Upright d in math mode (for differentials).
% -> d
\newcommand{\ud}{\mathrm{d}}

% Differential.
% -> dx
\newcommand{\diff}[1][x]{\,\ud{#1}}

% Base command for defining derivatives.
% -> df/dx or d^kf/dx^k
\newcommand{\basederivative}[4][]{%
  \displaystyle%
  \ifx\\#1\\\frac{#4#2}{#4#3}%
  \else%
  \frac{#4^#1#2}{#4#3^#1}%
  \fi%
}

% Total derivative.
% -> df/dx(x) or d^kf/dx^k(x)
\newcommand{\td}[4][]{%
  \basederivative[#1]{#2}{#3}{\ud}%
  \ifx\\#4\\%
  \else%
  \mkern-4mu\left(#4\right)%
  \fi%
}

% Partial derivative.
% -> df/dx(x) or d^kf/dx^k(x)
\newcommand{\pd}[4][]{%
  \basederivative[#1]{#2}{#3}{\partial}%
  \ifx\\#4\\%
  \else%
  \mkern-4mu\left(#4\right)%
  \fi%
}

\newcommand{\prob}[1]{\displaystyle P\left(#1\right)}

\newcommand{\pvalue}{\emph{$p$-value}}

\newcommand{\cond}{\,|\,}

\newcommand{\expect}[1]{\displaystyle E\left[#1\right]}

\newcommand{\mom}[2][]{\displaystyle {\cal M}_{#2}\ifx\\#1\\\else(#1)\fi}

\newcommand{\momalg}[1]{\displaystyle \lambda_{#1}}

\newcommand{\momcen}[1]{\displaystyle \mu_{#1}}

\newcommand{\skewness}{\displaystyle \gamma_1}

\newcommand{\kurtosis}{\displaystyle \gamma_2}

\newcommand{\charf}[1][x]{\phi_{#1}}

\newcommand{\momgenf}[1][x]{M_{#1}}

\newcommand{\fwhm}{{\scriptstyle \textsc{FWHM}}}

\newcommand{\hwhm}{{\scriptstyle \textsc{HWHM}}}

\newcommand{\median}{\mu_{\nicefrac{1}{2}}}

\newcommand{\var}[1]{\ensuremath{\text{Var}\left(#1\right)}}

\newcommand{\cov}[2]{\ensuremath{\text{Cov}\left(#1, #2\right)}}

\newcommand{\corr}[2]{\ensuremath{\text{Corr}\left(#1, #2\right)}}

\newcommand{\like}{\mathcal L}

\newcommand{\likelihood}[2][]{\like\ifx\\#2\\\else(#2\ifx\\#1\\\else;#1\fi)\fi}

\newcommand{\chisq}{\ensuremath{\chi^2}}

\newcommand{\chisquare}[2][]{\chisq\ifx\\#2\\\else(#2\ifx\\#1\\\else;#1\fi)\fi}

\newcommand{\loglikelihood}[2][]{\log\likelihood[#1]{#2}}

\newcommand{\pdf}[3][]{#2(#3\ifx\\#1\\\else;#1\fi)}

\newcommand{\binomialpdf}[2][]{\pdf[#1]{\mathcal B}{#2}}

\newcommand{\multinomialpdf}[2][]{\pdf[#1]{\mathcal M}{#2}}

\newcommand{\poissonpdf}[2][]{\pdf[#1]{\mathcal P}{#2}}

\newcommand{\uniformpdf}[2][]{\pdf[#1]{u}{#2}}

\newcommand{\exponentialpdf}[2][]{\pdf[#1]{\varepsilon}{#2}}

\newcommand{\gausspdf}[2][]{\pdf[#1]{N}{#2}}

\newcommand{\chisquarepdf}[2][]{\pdf[#1]{\wp}{#2}}

\newcommand{\cauchypdf}[2][]{\pdf[#1]{c}{#2}}

\newcommand{\erf}[1]{\ensuremath{\text{erf}\left(#1\right)}}

\newcommand{\dccases}[4][]{#2 \ifx\\#2\\\else=\fi %
  \begin{cases}
    \displaystyle #3 & \text{per variabili discrete}\\
    \displaystyle #4 & \text{per variabili continue}#1
  \end{cases}
}
% abbreviation for (sub_/super^)scripts of \lim, \sum,... in inline math
\newcommand{\ds}{\displaystyle}
\newcommand{\numberset}{\mathbb}
\newcommand{\C}{\numberset{C}}
\newcommand{\R}{\numberset{R}}
\newcommand{\Q}{\numberset{Q}}
\newcommand{\Z}{\numberset{Z}}
\newcommand{\N}{\numberset{N}}
\newcommand{\PP}{\mathbb{P}}
\DeclarePairedDelimiter{\norm}{\lVert}{\rVert}
\DeclarePairedDelimiter{\scalar}{\langle}{\rangle}

% Logarithm with arbitrary base.
% -> log_10
\newcommand{\llog}[1][10]{\log_{#1}}

% Absolute value.
% -> |x|
\newcommand{\abs}[1]{\left| #1 \right|}

% Powers.
% -> x^a
\newcommand{\power}[2][2]{\left( #2 \right)^{#1}}

% Square.
% -> x^2
\newcommand{\sq}[1]{\power[2]{#1}}

% Expansion of the binomial coefficient.
% -> n1!/(n2!(n1 - n2)!)
\newcommand{\binomexpr}[2]{\frac{#1!}{#2!(#1 - #2)!}}

% Expression evaluation at a given point with square brackets.
% -> [x]_{a}
\newcommand{\at}[2]{\left[ #1\right]_{\makebox[-1pt][l]{${\scriptstyle#2}$}}}

% Expression evaluation in an interval.
% -> [x] _{a}^{b}
\newcommand{\eval}[3]{\left.#1%
  \right|_{\makebox[-1pt][l]{${\scriptstyle#2}$}}^{\makebox[-1pt][l]{${\scriptstyle#3}$}}}

% Upright d in math mode (for differentials).
% -> d
\newcommand{\ud}{\mathrm{d}}

% Differential.
% -> dx
\newcommand{\diff}[1][x]{\,\ud{#1}}

% Base command for defining derivatives.
% -> df/dx or d^kf/dx^k
\newcommand{\basederivative}[4][]{%
  \displaystyle%
  \ifx\\#1\\\frac{#4#2}{#4#3}%
  \else%
  \frac{#4^#1#2}{#4#3^#1}%
  \fi%
}

% Total derivative.
% -> df/dx(x) or d^kf/dx^k(x)
\newcommand{\td}[4][]{%
  \basederivative[#1]{#2}{#3}{\ud}%
  \ifx\\#4\\%
  \else%
  \mkern-4mu\left(#4\right)%
  \fi%
}

% Partial derivative.
% -> df/dx(x) or d^kf/dx^k(x)
\newcommand{\pd}[4][]{%
  \basederivative[#1]{#2}{#3}{\partial}%
  \ifx\\#4\\%
  \else%
  \mkern-4mu\left(#4\right)%
  \fi%
}

\newcommand{\prob}[1]{\displaystyle P\left(#1\right)}

\newcommand{\pvalue}{\emph{$p$-value}}

\newcommand{\cond}{\,|\,}

\newcommand{\expect}[1]{\displaystyle E\left[#1\right]}

\newcommand{\mom}[2][]{\displaystyle {\cal M}_{#2}\ifx\\#1\\\else(#1)\fi}

\newcommand{\momalg}[1]{\displaystyle \lambda_{#1}}

\newcommand{\momcen}[1]{\displaystyle \mu_{#1}}

\newcommand{\skewness}{\displaystyle \gamma_1}

\newcommand{\kurtosis}{\displaystyle \gamma_2}

\newcommand{\charf}[1][x]{\phi_{#1}}

\newcommand{\momgenf}[1][x]{M_{#1}}

\newcommand{\fwhm}{{\scriptstyle \textsc{FWHM}}}

\newcommand{\hwhm}{{\scriptstyle \textsc{HWHM}}}

\newcommand{\median}{\mu_{\nicefrac{1}{2}}}

\newcommand{\var}[1]{\ensuremath{\text{Var}\left(#1\right)}}

\newcommand{\cov}[2]{\ensuremath{\text{Cov}\left(#1, #2\right)}}

\newcommand{\corr}[2]{\ensuremath{\text{Corr}\left(#1, #2\right)}}

\newcommand{\like}{\mathcal L}

\newcommand{\likelihood}[2][]{\like\ifx\\#2\\\else(#2\ifx\\#1\\\else;#1\fi)\fi}

\newcommand{\chisq}{\ensuremath{\chi^2}}

\newcommand{\chisquare}[2][]{\chisq\ifx\\#2\\\else(#2\ifx\\#1\\\else;#1\fi)\fi}

\newcommand{\loglikelihood}[2][]{\log\likelihood[#1]{#2}}

\newcommand{\pdf}[3][]{#2(#3\ifx\\#1\\\else;#1\fi)}

\newcommand{\binomialpdf}[2][]{\pdf[#1]{\mathcal B}{#2}}

\newcommand{\multinomialpdf}[2][]{\pdf[#1]{\mathcal M}{#2}}

\newcommand{\poissonpdf}[2][]{\pdf[#1]{\mathcal P}{#2}}

\newcommand{\uniformpdf}[2][]{\pdf[#1]{u}{#2}}

\newcommand{\exponentialpdf}[2][]{\pdf[#1]{\varepsilon}{#2}}

\newcommand{\gausspdf}[2][]{\pdf[#1]{N}{#2}}

\newcommand{\chisquarepdf}[2][]{\pdf[#1]{\wp}{#2}}

\newcommand{\cauchypdf}[2][]{\pdf[#1]{c}{#2}}

\newcommand{\erf}[1]{\ensuremath{\text{erf}\left(#1\right)}}

\newcommand{\dccases}[4][]{#2 \ifx\\#2\\\else=\fi %
  \begin{cases}
    \displaystyle #3 & \text{per variabili discrete}\\
    \displaystyle #4 & \text{per variabili continue}#1
  \end{cases}
}
\newcommand{\sgn}{\operatorname{sgn}}

\geometry{a4paper, left=30mm, right=30mm, top=30mm, bottom=30mm}


\title{Esercizi di Geometria Differenziale\\ del 16 Ottobre}
\author{Marco Romagnoli (578061)\thanks{svolti insieme a Bernardo Tomelleri (587829)}}
\date{\today}

\begin{document}
\maketitle

\section*{Esercizio 3.3}
Suppongo per assurdo che esistano $\overline{v}$ e $\overline{w}$ tali per cui $$\overline{v} \otimes \overline{w} = v \otimes w + v' \otimes w'.$$
Sapendo che $v$ e $v'$ sono indipendenti, così come $w$ e $w'$, considero due basi di $V$ 
\begin{align*}
\mathfrak{B}_1 &= \{ v_1 = v, v_2=v', v_3,\ldots,v_n\}\\
\mathfrak{B}_2 &= \{ w_1 = w, w_2 = w', w_3,\ldots	,w_n\}
\end{align*}
con le quali posso esprimere $\overline{v}$ e $\overline{w}$ come $$\overline{v} = \sum^n_{i=1} a_i v_i \qquad  \overline{w} = \sum^n_{j=1} b_j w_j$$
Quindi posso riscrivere
\begin{align*}
\overline{v} \otimes \overline{w} = \sum^n_{i=1} \sum^n_{j=1} a_i b_j v_i \otimes w_j
\end{align*}
ma, dato che i $\{v_i \otimes w_j\}$ formano una base per $\mathcal{T}^0_2$, i coefficienti devono essere $a_1b_1=a_2b_2=1$ e nulli tutti gli altri, in particolare $a_1b_2=a_2b_1=0$ che è chiaramente impossibile.\\
Quindi $v \otimes w + v' \otimes w'$ non può essere un elemento puro.

\section*{Esercizio 3.6}
\begin{proof}[Svolgimento]
Sia $\widetilde{T}$ un tensore di rango $(0,k)$ tale che $\widetilde{T}=T_S +T_A$ per un qualche coppia di tensori dello stesso rango $T_S$ simmetrico e $T_A$ antisimmetrico. Si può vedere facilmente che $$S(\widetilde{T}) = S(T_S) + S(T_A) = T_S$$ poiché un tensore simmetrico è uguale al suo simmetrizzato, mentre il simmetrizzato di un tensore antisimmetrico è nullo. Allo stesso modo si che $A(\widetilde{T}) = T_A$, quindi deve essere $$\widetilde{T}=S(\widetilde{T})+A(\widetilde{T}).$$
Se $k=2$ ogni tensore $T$ può essere espresso in questo modo, infatti 
\begin{align*}
T(v_1,v_2) &= \frac12 \left(T(v_1,v_2) + T(v_2,v_1)\right) + \frac12 \left(T(v_1,v_2) - T(v_2,v_1)\right) =\\
&= S(T)(v_1,v_2) + A(T)(v_1,v_2) \qquad \forall v_1,v_2 \in V
\end{align*}
Questo non vale per un generico $k\geq 3$, infatti si ha in generale 
\begin{align*}
S(T)(v_1,\ldots,v_k) + A(T)(v_1,\ldots,v_k) &= \frac{1}{k!}\sum_{\sigma\in S_k}{T(v_{\sigma(1)},\ldots,v_{\sigma(k)})} + \frac{1}{k!}\sum_{\sigma\in S_k}{\sgn{(\sigma)}T(v_{\sigma(1)},\ldots,v_{\sigma(k)})}=\\
&=\frac{2}{k!}\sum_{\sigma \;\text{pari}}{T(v_{\sigma(1)},\ldots,v_{\sigma(k)})} \neq T(v_1,\ldots,v_k)
\end{align*}
Se ad esempio prendo $T= e^1 \otimes e^2 \otimes \ldots \otimes e^k$, con $\{e^1,\ldots ,e^k\}$ base duale di $V^*$, a cui applico l'elemento $(e_{\tilde{\sigma}(1)},\ldots,e_{\tilde{\sigma}(k)})$, 
con $\{e_1,\ldots ,e_k\}$ base di $V$ e $\tilde{\sigma} \in S_k$ pari (diversa dall'identità), ottengo $0$, mentre se applico lo stesso elemento a $S(T) + A(T)$ ottengo $1$.
\end{proof}

\section*{Esercizio 3.7}
\begin{proof}[Svolgimento]
Sia $T$ un tensore di rango $(0,k)$. Se $T$ è antisimmetrico si ha per definizione che $$T(v_1,\ldots,v_i,\ldots,v_j,\ldots,v_k)=-T(v_1,\ldots,v_j,\ldots,v_i,\ldots,v_k)$$ per ogni coppa di indici $i\neq j$.
In particolare se $v_i=v_j=v$ si ha che 
\begin{align*}
&T(v_1,\ldots,v,\ldots,v,\ldots,v_k)=-T(v_1,\ldots,v,\ldots,v,\ldots,v_k)\\ &\Rightarrow T(v_1,\ldots,v,\ldots,v,\ldots,v_k) =0
\end{align*}
Se $T$ possiede la proprietà per cui $T(v_1,\ldots,v_k)=0$ se due qualsiasi vettori coincidono, si ha che vale
\begin{align*}
0 &=T(v_1,\ldots,v-w,\ldots,v-w,\ldots,v_k) =\\
&= -T(v_1,\ldots,v,\ldots,w,\ldots,v_k)-T(v_1,\ldots,w,\ldots,v,\ldots,v_k)\\
&\Rightarrow T(v_1,\ldots,v,\ldots,w,\ldots,v_k)=-T(v_1,\ldots,w,\ldots,v,\ldots,v_k)
\end{align*}
Quindi $T$ è antisimmetrico.\\

Se il simmetrizzato di $T$ è identicamente nullo, per la multilinearità di $T$ e il fatto che la cardinalità di $S_k$ è $k!$, si ha che:
\begin{align*}
S(T)(v,\ldots,v)=\frac{1}{k!}\sum_{\sigma\in S_k}{T(v,\ldots,v)}= \frac{k!}{k!}T\left(v,\ldots,v)= T(v,\ldots,v\right)=0 \quad \forall v\in V
\end{align*}
Al contrario, avendo $T$ la proprietà per cui $$T(v,\ldots,v)=0 \quad \forall v\in V$$ e prendendone il simmetrizzato si ha che 
\begin{align*}
S(T)(v_1,\ldots,v_k)=\frac{1}{k!}\sum_{\sigma\in S_k}{T(v_{\sigma(1)},\ldots,v_{\sigma(k)})}&=\\
= \frac{1}{k!}T\left((k-1)!\sum_{i=0}^k{v_i},\ldots,(k-1)!\sum_{i=0}^k{v_i}\right) &= 
 T\left(\sum_{i=0}^k{v_i},\ldots,\sum_{i=0}^k{v_i}\right)=0
\end{align*}

\end{proof}

\end{document}