\documentclass[italian,a4paper,10pt]{article}

% path to figure folder
\graphicspath{{figs/}}

\newtheorem{lem}{Lemma}[section]
\newtheorem{definizione}{Definizione}[section]
\newtheorem{teorema}{Teorema}[section]
\newtheorem{pro}{Proposizione}[section]
\newtheorem{cor}{Corollario}[section]
\declaretheoremstyle[
spaceabove=6pt, spacebelow=6pt,
headfont=\normalfont\bfseries,
notefont=\mdseries, notebraces={(}{)},
bodyfont=\normalfont,
postheadspace=1em,
qed=,
]{exercise_style}
\declaretheorem[style=exercise_style, name=Esempio, numberwithin=section]{example}
\declaretheorem[style=exercise_style, name=Esercizio, 
numberwithin=section]{exercise}
\declaretheorem[style=exercise_style, name=Problema,
numberwithin=subsection]{problem}
\declaretheorem[style=exercise_style, name=Osservazione, 
numberwithin=section]{observation}

% abbreviated labels lowercase c
\crefname{lem}{Lem.}{Lemm.}
\crefname{definizione}{Def.}{Def.ni}
\crefname{teorema}{Teo.}{Teo.mi}
\crefname{pro}{Prop.}{Propp.}
\crefname{cor}{Cor.}{Corr.}
% non-abbreviated labels uppercase C
\Crefname{lem}{Lemma}{Lemmi}
\Crefname{definizione}{Definizione}{Definizioni}
\Crefname{teorema}{Teorema}{Teoremi}
\Crefname{pro}{Proposizione}{Proposizioni}
\Crefname{cor}{Corollario}{Corollari}

% turns all (hyperlinked) references black [default is blue]
\hypersetup{
	linktoc=all,
	colorlinks=true,
	linkcolor=black
}

% imports math shorthand notation
% abbreviation for (sub_/super^)scripts of \lim, \sum,... in inline math
\newcommand{\ds}{\displaystyle}
\newcommand{\numberset}{\mathbb}
\newcommand{\C}{\numberset{C}}
\newcommand{\R}{\numberset{R}}
\newcommand{\Q}{\numberset{Q}}
\newcommand{\Z}{\numberset{Z}}
\newcommand{\N}{\numberset{N}}
\newcommand{\PP}{\mathbb{P}}
\DeclarePairedDelimiter{\norm}{\lVert}{\rVert}
\DeclarePairedDelimiter{\scalar}{\langle}{\rangle}

% Logarithm with arbitrary base.
% -> log_10
\newcommand{\llog}[1][10]{\log_{#1}}

% Absolute value.
% -> |x|
\newcommand{\abs}[1]{\left| #1 \right|}

% Powers.
% -> x^a
\newcommand{\power}[2][2]{\left( #2 \right)^{#1}}

% Square.
% -> x^2
\newcommand{\sq}[1]{\power[2]{#1}}

% Expansion of the binomial coefficient.
% -> n1!/(n2!(n1 - n2)!)
\newcommand{\binomexpr}[2]{\frac{#1!}{#2!(#1 - #2)!}}

% Expression evaluation at a given point with square brackets.
% -> [x]_{a}
\newcommand{\at}[2]{\left[ #1\right]_{\makebox[-1pt][l]{${\scriptstyle#2}$}}}

% Expression evaluation in an interval.
% -> [x] _{a}^{b}
\newcommand{\eval}[3]{\left.#1%
  \right|_{\makebox[-1pt][l]{${\scriptstyle#2}$}}^{\makebox[-1pt][l]{${\scriptstyle#3}$}}}

% Upright d in math mode (for differentials).
% -> d
\newcommand{\ud}{\mathrm{d}}

% Differential.
% -> dx
\newcommand{\diff}[1][x]{\,\ud{#1}}

% Base command for defining derivatives.
% -> df/dx or d^kf/dx^k
\newcommand{\basederivative}[4][]{%
  \displaystyle%
  \ifx\\#1\\\frac{#4#2}{#4#3}%
  \else%
  \frac{#4^#1#2}{#4#3^#1}%
  \fi%
}

% Total derivative.
% -> df/dx(x) or d^kf/dx^k(x)
\newcommand{\td}[4][]{%
  \basederivative[#1]{#2}{#3}{\ud}%
  \ifx\\#4\\%
  \else%
  \mkern-4mu\left(#4\right)%
  \fi%
}

% Partial derivative.
% -> df/dx(x) or d^kf/dx^k(x)
\newcommand{\pd}[4][]{%
  \basederivative[#1]{#2}{#3}{\partial}%
  \ifx\\#4\\%
  \else%
  \mkern-4mu\left(#4\right)%
  \fi%
}

\newcommand{\prob}[1]{\displaystyle P\left(#1\right)}

\newcommand{\pvalue}{\emph{$p$-value}}

\newcommand{\cond}{\,|\,}

\newcommand{\expect}[1]{\displaystyle E\left[#1\right]}

\newcommand{\mom}[2][]{\displaystyle {\cal M}_{#2}\ifx\\#1\\\else(#1)\fi}

\newcommand{\momalg}[1]{\displaystyle \lambda_{#1}}

\newcommand{\momcen}[1]{\displaystyle \mu_{#1}}

\newcommand{\skewness}{\displaystyle \gamma_1}

\newcommand{\kurtosis}{\displaystyle \gamma_2}

\newcommand{\charf}[1][x]{\phi_{#1}}

\newcommand{\momgenf}[1][x]{M_{#1}}

\newcommand{\fwhm}{{\scriptstyle \textsc{FWHM}}}

\newcommand{\hwhm}{{\scriptstyle \textsc{HWHM}}}

\newcommand{\median}{\mu_{\nicefrac{1}{2}}}

\newcommand{\var}[1]{\ensuremath{\text{Var}\left(#1\right)}}

\newcommand{\cov}[2]{\ensuremath{\text{Cov}\left(#1, #2\right)}}

\newcommand{\corr}[2]{\ensuremath{\text{Corr}\left(#1, #2\right)}}

\newcommand{\like}{\mathcal L}

\newcommand{\likelihood}[2][]{\like\ifx\\#2\\\else(#2\ifx\\#1\\\else;#1\fi)\fi}

\newcommand{\chisq}{\ensuremath{\chi^2}}

\newcommand{\chisquare}[2][]{\chisq\ifx\\#2\\\else(#2\ifx\\#1\\\else;#1\fi)\fi}

\newcommand{\loglikelihood}[2][]{\log\likelihood[#1]{#2}}

\newcommand{\pdf}[3][]{#2(#3\ifx\\#1\\\else;#1\fi)}

\newcommand{\binomialpdf}[2][]{\pdf[#1]{\mathcal B}{#2}}

\newcommand{\multinomialpdf}[2][]{\pdf[#1]{\mathcal M}{#2}}

\newcommand{\poissonpdf}[2][]{\pdf[#1]{\mathcal P}{#2}}

\newcommand{\uniformpdf}[2][]{\pdf[#1]{u}{#2}}

\newcommand{\exponentialpdf}[2][]{\pdf[#1]{\varepsilon}{#2}}

\newcommand{\gausspdf}[2][]{\pdf[#1]{N}{#2}}

\newcommand{\chisquarepdf}[2][]{\pdf[#1]{\wp}{#2}}

\newcommand{\cauchypdf}[2][]{\pdf[#1]{c}{#2}}

\newcommand{\erf}[1]{\ensuremath{\text{erf}\left(#1\right)}}

\newcommand{\dccases}[4][]{#2 \ifx\\#2\\\else=\fi %
  \begin{cases}
    \displaystyle #3 & \text{per variabili discrete}\\
    \displaystyle #4 & \text{per variabili continue}#1
  \end{cases}
}
% abbreviation for (sub_/super^)scripts of \lim, \sum,... in inline math
\newcommand{\ds}{\displaystyle}
\newcommand{\numberset}{\mathbb}
\newcommand{\C}{\numberset{C}}
\newcommand{\R}{\numberset{R}}
\newcommand{\Q}{\numberset{Q}}
\newcommand{\Z}{\numberset{Z}}
\newcommand{\N}{\numberset{N}}
\newcommand{\PP}{\mathbb{P}}
\DeclarePairedDelimiter{\norm}{\lVert}{\rVert}
\DeclarePairedDelimiter{\scalar}{\langle}{\rangle}

% Logarithm with arbitrary base.
% -> log_10
\newcommand{\llog}[1][10]{\log_{#1}}

% Absolute value.
% -> |x|
\newcommand{\abs}[1]{\left| #1 \right|}

% Powers.
% -> x^a
\newcommand{\power}[2][2]{\left( #2 \right)^{#1}}

% Square.
% -> x^2
\newcommand{\sq}[1]{\power[2]{#1}}

% Expansion of the binomial coefficient.
% -> n1!/(n2!(n1 - n2)!)
\newcommand{\binomexpr}[2]{\frac{#1!}{#2!(#1 - #2)!}}

% Expression evaluation at a given point with square brackets.
% -> [x]_{a}
\newcommand{\at}[2]{\left[ #1\right]_{\makebox[-1pt][l]{${\scriptstyle#2}$}}}

% Expression evaluation in an interval.
% -> [x] _{a}^{b}
\newcommand{\eval}[3]{\left.#1%
  \right|_{\makebox[-1pt][l]{${\scriptstyle#2}$}}^{\makebox[-1pt][l]{${\scriptstyle#3}$}}}

% Upright d in math mode (for differentials).
% -> d
\newcommand{\ud}{\mathrm{d}}

% Differential.
% -> dx
\newcommand{\diff}[1][x]{\,\ud{#1}}

% Base command for defining derivatives.
% -> df/dx or d^kf/dx^k
\newcommand{\basederivative}[4][]{%
  \displaystyle%
  \ifx\\#1\\\frac{#4#2}{#4#3}%
  \else%
  \frac{#4^#1#2}{#4#3^#1}%
  \fi%
}

% Total derivative.
% -> df/dx(x) or d^kf/dx^k(x)
\newcommand{\td}[4][]{%
  \basederivative[#1]{#2}{#3}{\ud}%
  \ifx\\#4\\%
  \else%
  \mkern-4mu\left(#4\right)%
  \fi%
}

% Partial derivative.
% -> df/dx(x) or d^kf/dx^k(x)
\newcommand{\pd}[4][]{%
  \basederivative[#1]{#2}{#3}{\partial}%
  \ifx\\#4\\%
  \else%
  \mkern-4mu\left(#4\right)%
  \fi%
}

\newcommand{\prob}[1]{\displaystyle P\left(#1\right)}

\newcommand{\pvalue}{\emph{$p$-value}}

\newcommand{\cond}{\,|\,}

\newcommand{\expect}[1]{\displaystyle E\left[#1\right]}

\newcommand{\mom}[2][]{\displaystyle {\cal M}_{#2}\ifx\\#1\\\else(#1)\fi}

\newcommand{\momalg}[1]{\displaystyle \lambda_{#1}}

\newcommand{\momcen}[1]{\displaystyle \mu_{#1}}

\newcommand{\skewness}{\displaystyle \gamma_1}

\newcommand{\kurtosis}{\displaystyle \gamma_2}

\newcommand{\charf}[1][x]{\phi_{#1}}

\newcommand{\momgenf}[1][x]{M_{#1}}

\newcommand{\fwhm}{{\scriptstyle \textsc{FWHM}}}

\newcommand{\hwhm}{{\scriptstyle \textsc{HWHM}}}

\newcommand{\median}{\mu_{\nicefrac{1}{2}}}

\newcommand{\var}[1]{\ensuremath{\text{Var}\left(#1\right)}}

\newcommand{\cov}[2]{\ensuremath{\text{Cov}\left(#1, #2\right)}}

\newcommand{\corr}[2]{\ensuremath{\text{Corr}\left(#1, #2\right)}}

\newcommand{\like}{\mathcal L}

\newcommand{\likelihood}[2][]{\like\ifx\\#2\\\else(#2\ifx\\#1\\\else;#1\fi)\fi}

\newcommand{\chisq}{\ensuremath{\chi^2}}

\newcommand{\chisquare}[2][]{\chisq\ifx\\#2\\\else(#2\ifx\\#1\\\else;#1\fi)\fi}

\newcommand{\loglikelihood}[2][]{\log\likelihood[#1]{#2}}

\newcommand{\pdf}[3][]{#2(#3\ifx\\#1\\\else;#1\fi)}

\newcommand{\binomialpdf}[2][]{\pdf[#1]{\mathcal B}{#2}}

\newcommand{\multinomialpdf}[2][]{\pdf[#1]{\mathcal M}{#2}}

\newcommand{\poissonpdf}[2][]{\pdf[#1]{\mathcal P}{#2}}

\newcommand{\uniformpdf}[2][]{\pdf[#1]{u}{#2}}

\newcommand{\exponentialpdf}[2][]{\pdf[#1]{\varepsilon}{#2}}

\newcommand{\gausspdf}[2][]{\pdf[#1]{N}{#2}}

\newcommand{\chisquarepdf}[2][]{\pdf[#1]{\wp}{#2}}

\newcommand{\cauchypdf}[2][]{\pdf[#1]{c}{#2}}

\newcommand{\erf}[1]{\ensuremath{\text{erf}\left(#1\right)}}

\newcommand{\dccases}[4][]{#2 \ifx\\#2\\\else=\fi %
  \begin{cases}
    \displaystyle #3 & \text{per variabili discrete}\\
    \displaystyle #4 & \text{per variabili continue}#1
  \end{cases}
}
\newcommand{\sgn}{\operatorname{sgn}}

\geometry{a4paper, left=30mm, right=30mm, top=30mm, bottom=30mm}


\title{Esercizi di Geometria Differenziale\\ del 18 Dicembre}
\author{Marco Romagnoli (578061)}
\date{\today}

\begin{document}
\maketitle

\section*{Esercizio 5.6}
\begin{proof}[Svolgimento]
Per ricavare il tensore metrico si usa la formula del cambio di coordinate dei tensori per passare dalle coordinate cartesiane $(x,y)$ a quelle polari $(\rho,\theta)$ $$g_{ij} = \pdv{x^a}{\overline{x}^i} \pdv{x^b}{\overline{x}^j}g^E_{ab}$$ dove $\bm{g}^E$ è il tensore metrico euclideo e $\overline{x}^i$ sono le coordinate polari, da cui, sapendo che $x= \rho\cos{\theta}$ e $y= \rho\sin{\theta}$, si ricava:
\begin{align*}
g_{11} &= \left(\pdv{x}{\rho}\right)^2+\left(\pdv{y}{\rho}\right)^2 = \cos^2{\theta}+\sin^2{\theta} = 1\\
g_{21} = g_{12} &= \pdv{x}{\rho}\pdv{x}{\theta} + \pdv{y}{\rho}\pdv{y}{\theta} = -\cos{\theta}\sin{\theta} + \cos{\theta}\sin{\theta} = 0\\
g_{22} &= \left(\pdv{x}{\theta}\right)^2+\left(\pdv{y}{\theta}\right)^2 = \rho^2\cos^2{\theta}+\rho^2\sin^2{\theta} = \rho^2
\end{align*}

Per ricavare i simboli di Christoffel a partire dal tensore metrico si può usare la formula$$\Gamma^k_{ij} = \frac12 g^{kl}\left(\pdv{g_{il}}{\overline{x}^j}+\pdv{g_{jl}}{\overline{x}^i}-\pdv{g_{ij}}{\overline{x}^l}\right)$$ da cui si ricava che
\begin{align*}
\Gamma^1_{11} &= \frac12 g^{11}\left(\pdv{g_{11}}{\rho}+\pdv{g_{11}}{\rho}-\pdv{g_{11}}{\rho}\right)=0\\
\Gamma^1_{12}=\Gamma^1_{21} &= \frac12 g^{11}\left(\pdv{g_{11}}{\theta}+\pdv{g_{21}}{\rho}-\pdv{g_{21}}{\rho}\right)=0\\
\Gamma^1_{22} &= \frac12 g^{11}\left(\pdv{g_{21}}{\theta}+\pdv{g_{21}}{\theta}-\pdv{g_{22}}{\rho}\right)=-\rho\\
\Gamma^2_{11} &= \frac12 g^{22}\left(\pdv{g_{12}}{\rho}+\pdv{g_{12}}{\rho}-\pdv{g_{11}}{\theta}\right)=0\\
\Gamma^2_{12}=\Gamma^2_{21} &= \frac12 g^{22}\left(\pdv{g_{22}}{\rho}+\pdv{g_{12}}{\theta}-\pdv{g_{21}}{\theta}\right)= \frac12 \rho^{-2} (2\rho) = \frac{1}{\rho}\\
\Gamma^2_{22} &= \frac12 g^{22}\left(\pdv{g_{}}{\theta}+\pdv{g_{22}}{\theta}-\pdv{g_{22}}{\theta}\right)=0
\end{align*} 
dove $g^{ij}$ sono le coordinate di 
\begin{equation*}
\bm{g}^{-1}= \begin{pmatrix}
1 & 0\\
0 & \rho^{-2}\\
\end{pmatrix}.
\end{equation*}

Da questo si può verificare che il tensore di Riemann è nullo sapendo che $$R_{ijk}^l = \pdv{\Gamma^l_{jk}}{x^i} - \pdv{\Gamma^l_{ik}}{x^j} + \Gamma^l_{im}\Gamma^n_{jk} - \Gamma^l_{jm}\Gamma^n_{ik}$$ e quindi esplicitando i conti (trascurando i termini in cui ogni addendo è nullo) si vede che:
\begin{align*}
R^1_{122} &= \pdv{\Gamma^1_{22}}{\rho} - \Gamma^1_{22}\Gamma^2_{12}= -1+\frac{\rho}{\rho}=0\\
R^1_{212} &= -\pdv{\Gamma^1_{22}}{\rho} + \Gamma^1_{22}\Gamma^2_{12}= 1-\frac{\rho}{\rho}=0\\
R^1_{221} &= \Gamma^1_{22}\Gamma^2_{12} - \Gamma^1_{22}\Gamma^2_{21} = 0\\
R^2_{112} &= \pdv{\Gamma^2_{12}}{\rho} - \pdv{\Gamma^2_{12}}{\rho} + \Gamma^2_{12}\Gamma^2_{21} -\Gamma^2_{12}\Gamma^2_{21} = 0\\
R^2_{121} &= \pdv{\Gamma^2_{21}}{\rho} + \Gamma^2_{12}\Gamma^2_{21}= -\frac{1}{\rho^2}+\frac{1}{\rho}\frac{1}{\rho}=0\\
R^2_{211} &= -\pdv{\Gamma^2_{21}}{\rho} - \Gamma^2_{12}\Gamma^2_{21}= \frac{1}{\rho^2}-\frac{1}{\rho}\frac{1}{\rho}=0
\end{align*}
\end{proof}

\section*{Esercizio 5.8}
\begin{proof}[Svolgimento]
Il piano iperbolico, in due dimensioni, è definito tramite il modello del semipiano come $$\HH^2 = \left\lbrace (x,y)\in \RR^2 | y>0 \right\rbrace $$ equipaggiato con il tensore metrico $\bm{g} = \frac{1}{y^2}\bm{g}_E$, dove $\bm{g}_E$ è il tensore metrico euclideo.
Tramite quest'ultimo, sapendo che $g^{ij}=y^2\delta^{ij}$ sono le coordinate di $\bm{g}^{-1}$, si possono calcolare esplicitamente i simboli di Christoffel:
\begin{align*}
\Gamma^1_{11} &= \frac12g^{11}\left(\pdv{g_{11}}{x}+\pdv{g_{11}}{x}-\pdv{g_{11}}{x}\right)=0\\
\Gamma^1_{12} = \Gamma^1_{21} &= \frac12g^{11}\left(\pdv{g_{11}}{y}+\pdv{g_{21}}{x}-\pdv{g_{12}}{x}\right)= \frac12y^2\left(-\frac{2}{y^3}\right)=-\frac1y\\ 
\Gamma^1_{22} &= \frac12g^{11}\left(\pdv{g_{21}}{x}+\pdv{g_{21}}{y}-\pdv{g_{22}}{x}\right)=0\\
\Gamma^2_{11} &= \frac12g^{22}\left(\pdv{g_{12}}{x}+\pdv{g_{12}}{x}-\pdv{g_{11}}{y}\right)=\frac12y^2\left(\frac{2}{y^3}\right)=\frac1y\\
\Gamma^2_{12} = \Gamma^2_{21} &= \frac12g^{22}\left(\pdv{g_{12}}{y}+\pdv{g_{22}}{x}-\pdv{g_{12}}{y}\right)=0\\
\Gamma^2_{22} &= \frac12g^{22}\left(\pdv{g_{22}}{y}+\pdv{g_{22}}{y}-\pdv{g_{22}}{y}\right)=\frac12y^2\left(-\frac{2}{y^3}\right)=-\frac1y\\
\end{align*}
Dal libro si può vedere che le geodetiche sono formate da  curve che hanno come supporto linee verticali e semicirconferenze ortogonali all'asse delle ascisse.
\end{proof}

\section*{Esercizio 5.10}
\begin{proof}[Svolgimento]
Data la connessione $\nabla$ su $\RR^3$ definita come nel testo dell'esercizio, si può vedere che $\Gamma^k_{ij}=\eps^{ijk}$ e quindi sicuramente non è simmetrico, infatti $\Gamma^k_{ij} = -\Gamma^k_{ji}$. 
Affinché sia compatibile con $\bm{g}$ deve essere vero che $\nabla_v \bm{g} = 0 \quad \forall v \in \RR^3$ che in carte diventa $$v^i \left(\pdv{g_{bc}}{x^i} - g_{jc}\Gamma^j_{ib} - g_{bj}\Gamma^j_{ic}\right)=0$$
In questo caso, dato che $g_{ij}=\delta_{ij}$ e in particolare è costante, si ha che
\begin{equation*}
g_{jc}\Gamma^j_{ib} + g_{bj}\Gamma^j_{ic} = \delta_{jc}\eps^{ibj} + \delta_{bj}\eps^{icj} = \eps^{ibc}+\eps^{icb} = \eps^{ibc}-\eps^{ibc} = 0
\end{equation*}
Quindi $\nabla$ è compatibile con il tensore metrico euclideo. 
Le geodetiche sono date dalle soluzioni dell'equazione $$D\dot{\bm{x}}= \ddot{\bm{x}} + \dot{x}^i\dot{x}^j\Gamma^k_{ij}\bm{e}_k=0$$
Si può vedere che il fattore $\dot{x}^i\dot{x}^j$ è simmetrico per scambio degli indici $i$ e $j$, mentre $\Gamma^k_{ij}$ è antisimmetrico per lo stesso scambio. Quindi $\dot{x}^i\dot{x}^j\Gamma^k_{ij}=0$ e l'equazione da risolvere è solamente $\ddot{\bm{x}}=0$, per cui le geodetiche sono curve del tipo $\bm{x}(t)=\bm{p}+\bm{v}t$ con $\bm{p},\bm{v} \in \RR^3$.
\end{proof}
\end{document}