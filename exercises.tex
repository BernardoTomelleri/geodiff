\documentclass{article}[a4paper, 12pt, italian]
\usepackage[utf8]{inputenc}
\usepackage{amsmath, amssymb, amsfonts, amscd, amsthm, thmtools, mathtools}
\usepackage[pdftex, hyperindex, plainpages=false]{hyperref}
\usepackage[nameinlink]{cleveref}
\usepackage{color}
\usepackage{graphicx}
\usepackage{nicefrac}
\usepackage{booktabs}
\usepackage{caption}
\usepackage{titling}
\usepackage{geometry}
\usepackage{float}
\usepackage{siunitx}
\usepackage[italian]{babel}
\usepackage{classicthesis}
\usepackage{enumitem}
\usepackage{tikz} %loads after classicthesis (xcolor incompat)
% path to figure folder
\graphicspath{{figs/}}

\newtheorem{lem}{Lemma}[section]
\newtheorem{definizione}{Definizione}[section]
\newtheorem{teorema}{Teorema}[section]
\newtheorem{pro}{Proposizione}[section]
\newtheorem{cor}{Corollario}[section]
\declaretheoremstyle[
spaceabove=6pt, spacebelow=6pt,
headfont=\normalfont\bfseries,
notefont=\mdseries, notebraces={(}{)},
bodyfont=\normalfont,
postheadspace=1em,
qed=,
]{exercise_style}
\declaretheorem[style=exercise_style, name=Esempio, numberwithin=section]{example}
\declaretheorem[style=exercise_style, name=Esercizio, 
numberwithin=section]{exercise}
\declaretheorem[style=exercise_style, name=Problema,
numberwithin=subsection]{problem}
\declaretheorem[style=exercise_style, name=Osservazione, 
numberwithin=section]{observation}

% abbreviated labels lowercase c
\crefname{lem}{Lem.}{Lemm.}
\crefname{definizione}{Def.}{Def.ni}
\crefname{teorema}{Teo.}{Teo.mi}
\crefname{pro}{Prop.}{Propp.}
\crefname{cor}{Cor.}{Corr.}
% non-abbreviated labels uppercase C
\Crefname{lem}{Lemma}{Lemmi}
\Crefname{definizione}{Definizione}{Definizioni}
\Crefname{teorema}{Teorema}{Teoremi}
\Crefname{pro}{Proposizione}{Proposizioni}
\Crefname{cor}{Corollario}{Corollari}

% turns all (hyperlinked) references black [default is blue]
\hypersetup{
	linktoc=all,
	colorlinks=true,
	linkcolor=black
}

% imports math shorthand notation
% abbreviation for (sub_/super^)scripts of \lim, \sum,... in inline math
\newcommand{\ds}{\displaystyle}
\newcommand{\numberset}{\mathbb}
\newcommand{\C}{\numberset{C}}
\newcommand{\R}{\numberset{R}}
\newcommand{\Q}{\numberset{Q}}
\newcommand{\Z}{\numberset{Z}}
\newcommand{\N}{\numberset{N}}
\newcommand{\PP}{\mathbb{P}}
\DeclarePairedDelimiter{\norm}{\lVert}{\rVert}
\DeclarePairedDelimiter{\scalar}{\langle}{\rangle}

% Logarithm with arbitrary base.
% -> log_10
\newcommand{\llog}[1][10]{\log_{#1}}

% Absolute value.
% -> |x|
\newcommand{\abs}[1]{\left| #1 \right|}

% Powers.
% -> x^a
\newcommand{\power}[2][2]{\left( #2 \right)^{#1}}

% Square.
% -> x^2
\newcommand{\sq}[1]{\power[2]{#1}}

% Expansion of the binomial coefficient.
% -> n1!/(n2!(n1 - n2)!)
\newcommand{\binomexpr}[2]{\frac{#1!}{#2!(#1 - #2)!}}

% Expression evaluation at a given point with square brackets.
% -> [x]_{a}
\newcommand{\at}[2]{\left[ #1\right]_{\makebox[-1pt][l]{${\scriptstyle#2}$}}}

% Expression evaluation in an interval.
% -> [x] _{a}^{b}
\newcommand{\eval}[3]{\left.#1%
  \right|_{\makebox[-1pt][l]{${\scriptstyle#2}$}}^{\makebox[-1pt][l]{${\scriptstyle#3}$}}}

% Upright d in math mode (for differentials).
% -> d
\newcommand{\ud}{\mathrm{d}}

% Differential.
% -> dx
\newcommand{\diff}[1][x]{\,\ud{#1}}

% Base command for defining derivatives.
% -> df/dx or d^kf/dx^k
\newcommand{\basederivative}[4][]{%
  \displaystyle%
  \ifx\\#1\\\frac{#4#2}{#4#3}%
  \else%
  \frac{#4^#1#2}{#4#3^#1}%
  \fi%
}

% Total derivative.
% -> df/dx(x) or d^kf/dx^k(x)
\newcommand{\td}[4][]{%
  \basederivative[#1]{#2}{#3}{\ud}%
  \ifx\\#4\\%
  \else%
  \mkern-4mu\left(#4\right)%
  \fi%
}

% Partial derivative.
% -> df/dx(x) or d^kf/dx^k(x)
\newcommand{\pd}[4][]{%
  \basederivative[#1]{#2}{#3}{\partial}%
  \ifx\\#4\\%
  \else%
  \mkern-4mu\left(#4\right)%
  \fi%
}

\newcommand{\prob}[1]{\displaystyle P\left(#1\right)}

\newcommand{\pvalue}{\emph{$p$-value}}

\newcommand{\cond}{\,|\,}

\newcommand{\expect}[1]{\displaystyle E\left[#1\right]}

\newcommand{\mom}[2][]{\displaystyle {\cal M}_{#2}\ifx\\#1\\\else(#1)\fi}

\newcommand{\momalg}[1]{\displaystyle \lambda_{#1}}

\newcommand{\momcen}[1]{\displaystyle \mu_{#1}}

\newcommand{\skewness}{\displaystyle \gamma_1}

\newcommand{\kurtosis}{\displaystyle \gamma_2}

\newcommand{\charf}[1][x]{\phi_{#1}}

\newcommand{\momgenf}[1][x]{M_{#1}}

\newcommand{\fwhm}{{\scriptstyle \textsc{FWHM}}}

\newcommand{\hwhm}{{\scriptstyle \textsc{HWHM}}}

\newcommand{\median}{\mu_{\nicefrac{1}{2}}}

\newcommand{\var}[1]{\ensuremath{\text{Var}\left(#1\right)}}

\newcommand{\cov}[2]{\ensuremath{\text{Cov}\left(#1, #2\right)}}

\newcommand{\corr}[2]{\ensuremath{\text{Corr}\left(#1, #2\right)}}

\newcommand{\like}{\mathcal L}

\newcommand{\likelihood}[2][]{\like\ifx\\#2\\\else(#2\ifx\\#1\\\else;#1\fi)\fi}

\newcommand{\chisq}{\ensuremath{\chi^2}}

\newcommand{\chisquare}[2][]{\chisq\ifx\\#2\\\else(#2\ifx\\#1\\\else;#1\fi)\fi}

\newcommand{\loglikelihood}[2][]{\log\likelihood[#1]{#2}}

\newcommand{\pdf}[3][]{#2(#3\ifx\\#1\\\else;#1\fi)}

\newcommand{\binomialpdf}[2][]{\pdf[#1]{\mathcal B}{#2}}

\newcommand{\multinomialpdf}[2][]{\pdf[#1]{\mathcal M}{#2}}

\newcommand{\poissonpdf}[2][]{\pdf[#1]{\mathcal P}{#2}}

\newcommand{\uniformpdf}[2][]{\pdf[#1]{u}{#2}}

\newcommand{\exponentialpdf}[2][]{\pdf[#1]{\varepsilon}{#2}}

\newcommand{\gausspdf}[2][]{\pdf[#1]{N}{#2}}

\newcommand{\chisquarepdf}[2][]{\pdf[#1]{\wp}{#2}}

\newcommand{\cauchypdf}[2][]{\pdf[#1]{c}{#2}}

\newcommand{\erf}[1]{\ensuremath{\text{erf}\left(#1\right)}}

\newcommand{\dccases}[4][]{#2 \ifx\\#2\\\else=\fi %
  \begin{cases}
    \displaystyle #3 & \text{per variabili discrete}\\
    \displaystyle #4 & \text{per variabili continue}#1
  \end{cases}
}

% adjustable page margins, currently scientific article standards
\geometry{left=25mm, right=25mm, top=25mm, bottom=25mm}

\title{Esercizi di Geometria differenziale}
\author{Bernardo Tomelleri\thanks{Università di Pisa}}
\date{\today}

\begin{document}
\maketitle

\section{Esercizi del 02/10/2021}
\begin{exercise}
Siano $X$ e $Y$ due spazi topologici. La topologia prodotto su $X \times Y$
è definita nel modo seguente: un sottoinsieme $A \subseteq X \times Y$ è
aperto se e solo se è unione arbitraria di sottoinsiemi $U \times V$ dove
$U \subseteq X$ e $V \subseteq Y$ sono entrambi aperti. Mostra che questa è
veramente una topologia su $X \times Y$.
\begin{proof}[Svolgimento]

\end{proof}
\end{exercise}

\begin{exercise}
Sia $f: X \to Y$ una funzione suriettiva da uno spazio topologico $X$ su
un insieme $Y.$ La topologia quoziente su $Y$ è definita nel modo seguente:
un sottoinsieme $A \subseteq Y$ è aperto se e solo se la sua controimmagine
$f^{-1}(A)$ è aperta. Mostra che questa è veramente una topologia su $X$.
\begin{proof}[Svolgimento]

\end{proof}
\end{exercise}

\begin{exercise}
Sia $f: X \to Y$ una funzione fra spazi topologici. Mostra che $f$ è
continua se e solo se vale il fatto seguente: per ogni $x \in X$ e per ogni
intorno $A$ di $f(x)$, la controimmagine $f^{-1}(A)$ è un intorno di $x$.
\begin{proof}[Svolgimento]

\end{proof}
\end{exercise}


\begin{exercise}
Sia $K$ uno spazio topologico compatto. Sia $C \subseteq K$ un sottoinsieme
chiuso. Mostra che $C$ è compatto.
\begin{proof}[Svolgimento]
Per ipotesi esiste un ricoprimento aperto finito $\left\{A_i\right\}$ tale
che $\bigcup_i A_i = K$. Se $C \subseteq K$, allora lo stesso ricoprimento
a maggior ragione copre anche $C$ e continua ad essere finito.
\end{proof}
\end{exercise}

\begin{exercise}
Mostra che il segmento $[0, 1]$ è connesso, usando solo la definizione di
connesso (e nessun altro teorema: di solito questo fatto si mostra subito dopo
la definizione).
\begin{proof}
Per assurdo supponiamo che $[0, 1]$ sia unione disgiunta di due sottoinsiemi
aperti $A, B \subseteq [0, 1] \; : A \cup B = [0, 1]$ con
$A \neq \emptyset$, $B \neq \emptyset$ e $A \cap B = \emptyset$.
Supponiamo che $0 \in A$, poiché $A$ è aperto $\exists \epsilon > 0$
tale che un intorno $U(0)_\epsilon = [0, \epsilon) \subseteq A$.
Consideriamo il $\sup_\epsilon \{U(0)_\epsilon\} = U(0)_\eta$ di questi
intorni; per ipotesi dev'essere $\eta \neq 1$ (altrimenti $[0, 1)$ sarebbe
contenuto in $A$, dunque $A = [0, 1]$ chiuso). Ora, poiché 
$A \cap B = \emptyset$ sono aperti disgiunti, $\eta \not\in B$, ma
$\eta \in A$. Ma allora sempre per apertura di $A$ dovrebbe esistere un
intorno di $\eta$ contenuto in $[0, 1)$ e quindi un secondo raggio 
$\eta' > \eta$ per cui vale ancora $U(0)_\eta' = [0, \eta') \subseteq A$.
Assurdo per definizione di $\eta$ come $\sup$. Da cui concludiamo che
$\eta \in A$, $A = [0, 1]$ e $B = \emptyset$, cioè $[0, 1]$ non è
sconnesso.

\end{proof}
\end{exercise}

\begin{exercise}
Mostra che il sottoinsieme seguente in $\R^2$ è connesso ma non connesso per
archi:
\[
X = \left\{(0, y) | y \in [-1, 1]\right\} \bigcup
\left\{(x, \sin{\nicefrac{1}{x}}) | x > 0\right\}
.\]
\begin{proof}[Svolgimento]

\end{proof}
\end{exercise}

\begin{exercise}
Scrivere le funzioni di transizione di uno dei due atlanti che abbiamo scelto
per $S^n$ e verifica che sono lisce.
\begin{proof}[Svolgimento]

\end{proof}
\end{exercise}

\begin{exercise}
Mostra che la mappa
\[
f: S^n \to \R\PP^n, (x_1, x_2, \ldots, x_{n+1} \mapsto
[x_1, x_2, \ldots, x_{n+1}]
\]
è liscia.
\begin{proof}[Svolgimento]

\end{proof}
\end{exercise}

Un \emph{diffeomorfismo} è una mappa liscia $f: M \to N$ fra varietà lisce
che ha un'inversa, anch'essa liscia.

\begin{exercise}
Costruisci due atlanti \emph{non} compatibili per la varietà topologica $\R$.
Mostra però che le due varietà lisce risultanti sono comunque diffeomorfe!
\begin{proof}[Svolgimento]

\end{proof}
\end{exercise}

\begin{exercise}
Mostra che $\R\PP^1$ e $S^1$ sono diffeomorfi.
\begin{proof}[Svolgimento]

\end{proof}
\end{exercise}
\end{document}
