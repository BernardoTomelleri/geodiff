\documentclass{article}[a4paper, 12pt, italian]
\usepackage[utf8]{inputenc}
\usepackage{amsmath, amssymb, amsfonts, amscd, amsthm, thmtools, mathtools}
\usepackage[pdftex, hyperindex, plainpages=false]{hyperref}
\usepackage[nameinlink]{cleveref}
\usepackage{color}
\usepackage{graphicx}
\usepackage{nicefrac}
\usepackage{booktabs}
\usepackage{caption}
\usepackage{titling}
\usepackage{geometry}
\usepackage{float}
\usepackage{siunitx}
\usepackage[italian]{babel}
\usepackage{classicthesis}
\usepackage{enumitem}
\usepackage{tikz} %loads after classicthesis (xcolor incompat)
% path to figure folder
\graphicspath{{figs/}}

\newtheorem{lem}{Lemma}[section]
\newtheorem{definizione}{Definizione}[section]
\newtheorem{teorema}{Teorema}[section]
\newtheorem{pro}{Proposizione}[section]
\newtheorem{cor}{Corollario}[section]
\declaretheoremstyle[
spaceabove=6pt, spacebelow=6pt,
headfont=\normalfont\bfseries,
notefont=\mdseries, notebraces={(}{)},
bodyfont=\normalfont,
postheadspace=1em,
qed=,
]{exercise_style}
\declaretheorem[style=exercise_style, name=Esempio, numberwithin=section]{example}
\declaretheorem[style=exercise_style, name=Esercizio, 
numberwithin=section]{exercise}
\declaretheorem[style=exercise_style, name=Problema,
numberwithin=subsection]{problem}
\declaretheorem[style=exercise_style, name=Osservazione, 
numberwithin=section]{observation}

% abbreviated labels lowercase c
\crefname{lem}{Lem.}{Lemm.}
\crefname{definizione}{Def.}{Def.ni}
\crefname{teorema}{Teo.}{Teo.mi}
\crefname{pro}{Prop.}{Propp.}
\crefname{cor}{Cor.}{Corr.}
% non-abbreviated labels uppercase C
\Crefname{lem}{Lemma}{Lemmi}
\Crefname{definizione}{Definizione}{Definizioni}
\Crefname{teorema}{Teorema}{Teoremi}
\Crefname{pro}{Proposizione}{Proposizioni}
\Crefname{cor}{Corollario}{Corollari}

% turns all (hyperlinked) references black [default is blue]
\hypersetup{
	linktoc=all,
	colorlinks=true,
	linkcolor=black
}

% imports math shorthand notation
\declaretheoremstyle[
spaceabove=6pt, spacebelow=6pt,
headfont=\normalfont\bfseries\itshape,
notefont=\mdseries, notebraces={(}{)},
bodyfont=\normalfont,
postheadspace=1em,
qed=,
]{exercise_style}

\declaretheoremstyle[
spaceabove=6pt, spacebelow=6pt,
postheadspace=1em,
qed=,
]{theorem_style}

\declaretheoremstyle[
spaceabove=6pt, spacebelow=6pt,
postheadspace=1em,
qed=,
]{axiom_style}

\declaretheorem[name=Teorema,numberwithin=section]{theorem}
\declaretheorem[name=Lemma,sibling=theorem]{lemma}
\declaretheorem[name=Proposizione,sibling=theorem]{proposition}
\declaretheorem[name=Corollario,sibling=theorem]{corollary}
\declaretheorem[name=Paradosso,sibling=theorem]{paradox}
\declaretheorem[style=axiom_style,name=Assioma,sibling=theorem]{axiom}
\declaretheorem[name=Definizione,numberwithin=section]{definition}
\declaretheorem[style=exercise_style,name=Esempio,numberwithin=section]{example}
\declaretheorem[style=exercise_style,name=Esercizio,numberwithin=section]{exercise}
\declaretheorem[style=exercise_style,name=Osservazione,numberwithin=section]{remark}

%\renewcommand\qedsymbol{Deh, per forza!}

%\newcommand{\abs}[1]{{\left|#1\right|}}
%\newcommand{\norm}[1]{{\|#1\|}}
\DeclareMathOperator{\Imaginarypart}{Im}
\renewcommand{\Im}{\Imaginarypart}
\DeclareMathOperator{\Realpart}{Re}
\renewcommand{\Re}{\Realpart}

% greek letters
\newcommand{\eps}{\varepsilon}
\renewcommand{\phi}{\varphi}

% blackboard letters
\newcommand{\CC}{\mathbb C}
\newcommand{\HH}{\mathbb H}
\newcommand{\II}{\mathbb{I}}
\newcommand{\KK}{\mathbb K}
\newcommand{\NN}{\mathbb N}
\newcommand{\QQ}{\mathbb Q}
\newcommand{\RR}{\mathbb R}
\newcommand{\TT}{\mathbb T}
\newcommand{\ZZ}{\mathbb Z}

% Upright d in math mode (for differentials).
% -> d
\newcommand{\ud}{\mathrm{d}}

% Differential.
% -> dx
\newcommand{\diff}[1][x]{\,\ud{#1}}

% Base command for defining derivatives.
% -> df/dx or d^kf/dx^k
\newcommand{\basederivative}[4][]{%
  \displaystyle%
  \ifx\\#1\\\frac{#4#2}{#4#3}%
  \else%
  \frac{#4^#1#2}{#4#3^#1}%
  \fi%
}

% Total derivative.
% -> df/dx(x) or d^kf/dx^k(x)
\newcommand{\td}[4][]{%
  \basederivative[#1]{#2}{#3}{\ud}%
  \ifx\\#4\\%
  \else%
  \mkern-4mu\left(#4\right)%
  \fi%
}

% Partial derivative.
% -> df/dx(x) or d^kf/dx^k(x)
\newcommand{\pd}[4][]{%
  \basederivative[#1]{#2}{#3}{\partial}%
  \ifx\\#4\\%
  \else%
  \mkern-4mu\left(#4\right)%
  \fi%
}

% Tilde under variables v_~
\DeclareMathAccent{\wtilde}{\mathord}{largesymbols}{"65}
\newcommand{\utld}[1]{\underaccent{\wtilde}{#1}}

\newcommand{\ds}{\displaystyle}

% Image of a map
\DeclareMathOperator{\Ima}{Im}


% adjustable page margins, currently scientific article standards
\geometry{left=25mm, right=25mm, top=25mm, bottom=25mm}

\title{Esercizi di Geometria differenziale}
\author{Bernardo Tomelleri\thanks{Università di Pisa}}
\date{\today}

\begin{document}
\maketitle

\section{Esercizi del 02/10/2021}
\begin{exercise}
Siano $X$ e $Y$ due spazi topologici. La topologia prodotto su $X \times Y$
è definita nel modo seguente: un sottoinsieme $A \subseteq X \times Y$ è
aperto se e solo se è unione arbitraria di sottoinsiemi $U \times V$ dove
$U \subseteq X$ e $V \subseteq Y$ sono entrambi aperti. Mostra che questa è
veramente una topologia su $X \times Y$.
\begin{proof}[Svolgimento]

\end{proof}
\end{exercise}

\begin{exercise}
Sia $f: X \to Y$ una funzione suriettiva da uno spazio topologico $X$ su
un insieme $Y.$ La topologia quoziente su $Y$ è definita nel modo seguente:
un sottoinsieme $A \subseteq Y$ è aperto se e solo se la sua controimmagine
$f^{-1}(A)$ è aperta. Mostra che questa è veramente una topologia su $X$.
\begin{proof}[Svolgimento]

\end{proof}
\end{exercise}

\begin{exercise}
Sia $f: X \to Y$ una funzione fra spazi topologici. Mostra che $f$ è
continua se e solo se vale il fatto seguente: per ogni $x \in X$ e per ogni
intorno $A$ di $f(x)$, la controimmagine $f^{-1}(A)$ è un intorno di $x$.
\begin{proof}[Svolgimento]

\end{proof}
\end{exercise}


\begin{exercise}
Sia $K$ uno spazio topologico compatto. Sia $C \subseteq K$ un sottoinsieme
chiuso. Mostra che $C$ è compatto.
\begin{proof}[Svolgimento]

\end{proof}
\end{exercise}

\begin{exercise}
Mostra che il segmento $[0, 1]$ è connesso, usando solo la definizione di
connesso (e nessun altro teorema: di solito questo fatto si mostra subito dopo
la definizione).
\begin{proof}[Svolgimento]

\end{proof}
\end{exercise}

\begin{exercise}
Mostra che il sottoinsieme seguente in $\R^2$ è connesso ma non connesso per
archi:
\[
X = \left\{(0, y) | y \in [-1, 1]\right\} \bigcup
\left\{(x, \sin{\nicefrac{1}{x}}) | x > 0\right\}
.\]
\begin{proof}[Svolgimento]

\end{proof}
\end{exercise}

\begin{exercise}
Scrivere le funzioni di transizione di uno dei due atlanti che abbiamo scelto
per $S^n$ e verifica che sono lisce.
\begin{proof}[Svolgimento]

\end{proof}
\end{exercise}

\begin{exercise}
Mostra che la mappa
\[
f: S^n \to \R\PP^n, (x_1, x_2, \ldots, x_{n+1} \mapsto
[x_1, x_2, \ldots, x_{n+1}]
\]
è liscia.
\begin{proof}[Svolgimento]

\end{proof}
\end{exercise}

Un \emph{diffeomorfismo} è una mappa liscia $f: M \to N$ fra varietà lisce
che ha un'inversa, anch'essa liscia.

\begin{exercise}
Costruisci due atlanti \emph{non} compatibili per la varietà topologica $\R$.
Mostra però che le due varietà lisce risultanti sono comunque diffeomorfe!
\begin{proof}[Svolgimento]

\end{proof}
\end{exercise}

\begin{exercise}
Mostra che $\R\PP^1$ e $S^1$ sono diffeomorfi.
\begin{proof}[Svolgimento]

\end{proof}
\end{exercise}
\end{document}
