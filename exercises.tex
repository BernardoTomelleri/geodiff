\documentclass[a4paper, 12pt, italian]{article}
\usepackage[utf8]{inputenc}
\usepackage{amsmath, amssymb, amsfonts, amscd, amsthm, thmtools, mathtools}
\usepackage[pdftex, hyperindex, plainpages=false]{hyperref}
\usepackage[nameinlink]{cleveref}
\usepackage{color}
\usepackage{graphicx}
\usepackage{nicefrac}
\usepackage{booktabs}
\usepackage{caption}
\usepackage{titling}
\usepackage{geometry}
\usepackage{float}
\usepackage{siunitx}
\usepackage[italian]{babel}
\usepackage{classicthesis}
\usepackage{enumitem}
\usepackage{tikz} %loads after classicthesis (xcolor incompat)
% path to figure folder
\graphicspath{{figs/}}

\newtheorem{lem}{Lemma}[section]
\newtheorem{definizione}{Definizione}[section]
\newtheorem{teorema}{Teorema}[section]
\newtheorem{pro}{Proposizione}[section]
\newtheorem{cor}{Corollario}[section]
\declaretheoremstyle[
spaceabove=6pt, spacebelow=6pt,
headfont=\normalfont\bfseries,
notefont=\mdseries, notebraces={(}{)},
bodyfont=\normalfont,
postheadspace=1em,
qed=,
]{exercise_style}
\declaretheorem[style=exercise_style, name=Esempio, numberwithin=section]{example}
\declaretheorem[style=exercise_style, name=Esercizio, 
numberwithin=section]{exercise}
\declaretheorem[style=exercise_style, name=Problema,
numberwithin=subsection]{problem}
\declaretheorem[style=exercise_style, name=Osservazione, 
numberwithin=section]{observation}

% abbreviated labels lowercase c
\crefname{lem}{Lem.}{Lemm.}
\crefname{definizione}{Def.}{Def.ni}
\crefname{teorema}{Teo.}{Teo.mi}
\crefname{pro}{Prop.}{Propp.}
\crefname{cor}{Cor.}{Corr.}
% non-abbreviated labels uppercase C
\Crefname{lem}{Lemma}{Lemmi}
\Crefname{definizione}{Definizione}{Definizioni}
\Crefname{teorema}{Teorema}{Teoremi}
\Crefname{pro}{Proposizione}{Proposizioni}
\Crefname{cor}{Corollario}{Corollari}

% turns all (hyperlinked) references black [default is blue]
\hypersetup{
	linktoc=all,
	colorlinks=true,
	linkcolor=black
}

% imports math shorthand notation
\declaretheoremstyle[
spaceabove=6pt, spacebelow=6pt,
headfont=\normalfont\bfseries\itshape,
notefont=\mdseries, notebraces={(}{)},
bodyfont=\normalfont,
postheadspace=1em,
qed=,
]{exercise_style}

\declaretheoremstyle[
spaceabove=6pt, spacebelow=6pt,
postheadspace=1em,
qed=,
]{theorem_style}

\declaretheoremstyle[
spaceabove=6pt, spacebelow=6pt,
postheadspace=1em,
qed=,
]{axiom_style}

\declaretheorem[name=Teorema,numberwithin=section]{theorem}
\declaretheorem[name=Lemma,sibling=theorem]{lemma}
\declaretheorem[name=Proposizione,sibling=theorem]{proposition}
\declaretheorem[name=Corollario,sibling=theorem]{corollary}
\declaretheorem[name=Paradosso,sibling=theorem]{paradox}
\declaretheorem[style=axiom_style,name=Assioma,sibling=theorem]{axiom}
\declaretheorem[name=Definizione,numberwithin=section]{definition}
\declaretheorem[style=exercise_style,name=Esempio,numberwithin=section]{example}
\declaretheorem[style=exercise_style,name=Esercizio,numberwithin=section]{exercise}
\declaretheorem[style=exercise_style,name=Osservazione,numberwithin=section]{remark}

%\renewcommand\qedsymbol{Deh, per forza!}

%\newcommand{\abs}[1]{{\left|#1\right|}}
%\newcommand{\norm}[1]{{\|#1\|}}
\DeclareMathOperator{\Imaginarypart}{Im}
\renewcommand{\Im}{\Imaginarypart}
\DeclareMathOperator{\Realpart}{Re}
\renewcommand{\Re}{\Realpart}

% greek letters
\newcommand{\eps}{\varepsilon}
\renewcommand{\phi}{\varphi}

% blackboard letters
\newcommand{\CC}{\mathbb C}
\newcommand{\HH}{\mathbb H}
\newcommand{\II}{\mathbb{I}}
\newcommand{\KK}{\mathbb K}
\newcommand{\NN}{\mathbb N}
\newcommand{\QQ}{\mathbb Q}
\newcommand{\RR}{\mathbb R}
\newcommand{\TT}{\mathbb T}
\newcommand{\ZZ}{\mathbb Z}

% Upright d in math mode (for differentials).
% -> d
\newcommand{\ud}{\mathrm{d}}

% Differential.
% -> dx
\newcommand{\diff}[1][x]{\,\ud{#1}}

% Base command for defining derivatives.
% -> df/dx or d^kf/dx^k
\newcommand{\basederivative}[4][]{%
  \displaystyle%
  \ifx\\#1\\\frac{#4#2}{#4#3}%
  \else%
  \frac{#4^#1#2}{#4#3^#1}%
  \fi%
}

% Total derivative.
% -> df/dx(x) or d^kf/dx^k(x)
\newcommand{\td}[4][]{%
  \basederivative[#1]{#2}{#3}{\ud}%
  \ifx\\#4\\%
  \else%
  \mkern-4mu\left(#4\right)%
  \fi%
}

% Partial derivative.
% -> df/dx(x) or d^kf/dx^k(x)
\newcommand{\pd}[4][]{%
  \basederivative[#1]{#2}{#3}{\partial}%
  \ifx\\#4\\%
  \else%
  \mkern-4mu\left(#4\right)%
  \fi%
}

% Tilde under variables v_~
\DeclareMathAccent{\wtilde}{\mathord}{largesymbols}{"65}
\newcommand{\utld}[1]{\underaccent{\wtilde}{#1}}

\newcommand{\ds}{\displaystyle}

% Image of a map
\DeclareMathOperator{\Ima}{Im}


% adjustable page margins, currently scientific article standards
\geometry{left=25mm, right=25mm, top=25mm, bottom=25mm}

\title{Esercizi di Geometria differenziale}
\author{Bernardo Tomelleri\thanks{Esercizi svolti in collaborazione con Marco
Romagnoli $(578061)$} $(587829)$}
\date{\today}

\begin{document}
\maketitle

\section{Esercizi del 02/10/2021}
\begin{exercise}
Siano $X$ e $Y$ due spazi topologici. La topologia prodotto su $X \times Y$
è definita nel modo seguente: un sottoinsieme $A \subseteq X \times Y$ è
aperto se e solo se è unione arbitraria di sottoinsiemi $U \times V$ dove
$U \subseteq X$ e $V \subseteq Y$ sono entrambi aperti. Mostra che questa è
veramente una topologia su $X \times Y$.
\begin{proof}[Svolgimento]
Sicuramente l'insieme vuoto $\emptyset$ e l'intero insieme $X \times Y$ sono
aperti nella topologia prodotto, visto che si possono scrivere come prodotto
di sottoinsiemi aperti di $X$ e $Y$. Basta prendere come sottoinsiemi aperti
gli insiemi vuoti $\emptyset_X$ e $\emptyset_Y$ degli spazi di partenza e i
sottoinsiemi interi $X$ e $Y$ aperti per definizione negli spazi topologici
$X$ e $Y$.

\end{proof}
\end{exercise}

\begin{exercise}
Sia $f: X \to Y$ una funzione suriettiva da uno spazio topologico $X$ su
un insieme $Y.$ La topologia quoziente su $Y$ è definita nel modo seguente:
un sottoinsieme $A \subseteq Y$ è aperto se e solo se la sua controimmagine
$f^{-1}(A)$ è aperta. Mostra che questa è veramente una topologia su $X$.
\begin{proof}[Svolgimento]
Osserviamo che la controimmagine dell'insieme vuoto $\emptyset$ e
dell'insieme delle classi di equivalenza $Y$ definite da $f$ sono
rispettivamente $f^{-1}(\emptyset) = \emptyset$ e $f^{-1}(Y) = X$, che
sono entrambi aperti in $X$ per definizione di spazio topologico.
 
\end{proof}
\end{exercise}

\begin{exercise}
Sia $f: X \to Y$ una funzione fra spazi topologici. Mostra che $f$ è
continua se e solo se vale il fatto seguente: per ogni $x \in X$ e per ogni
intorno $A$ di $f(x)$, la controimmagine $f^{-1}(A)$ è un intorno di $x$.
\begin{proof}[Svolgimento]
Per definizione $f$ è continua se la controimmagine di ogni sottoinsieme
aperto di $Y$ è un aperto in $X$:
Se $A \in \tau_Y \implies f^{-1}(A) \in \tau_X$. Per prima cosa supponiamo
$f$ continua e prendiamo un generico punto $x \in X$ e un intorno $A$ di $f(x)$
in $Y$. Per ipotesi $f^{-1}(A)$ è un aperto di $X$ che contiene $x$ (visto
che per costruzione $f(x) \in A$) quindi è un intorno di $x$.\newline
Viceversa, se $\forall x \in X$ e $\forall A$ intorno di $f(x)$ in $Y$ la sua
controimmagine $f^{-1}(A) \subseteq X$ è un intorno di $x$, consideriamo un
aperto $B \subseteq Y$ qualsiasi:
\begin{enumerate}
\item Se $B \cap f(X) = \emptyset$, allora $f^{-1}(B) = \emptyset$ che per
definizione di spazio topologico è sempre un aperto di $X$.
\item Se invece $B \cap f(X) \neq \emptyset \implies \exists x: f(x) \in B$,
cioè $B$ è un intorno di $f(x)$. Per ipotesi allora $f^{-1}(B) \subseteq X$ 
è a sua volta un intorno di $x$ e, a maggior ragione, $f^{-1}(B)$ è un
aperto di $X$.
\end{enumerate}
\end{proof}
\end{exercise}


\begin{exercise}
Sia $K$ uno spazio topologico compatto. Sia $C \subseteq K$ un sottoinsieme
chiuso. Mostra che $C$ è compatto.
\begin{proof}[Svolgimento]
Per ipotesi tutti i ricoprimenti aperti di $K$, $\left\{A_i\right\}_{i \in I} : K \subseteq \bigcup_{i \in I} A_i$ ammettono un sottoricoprimento
finito $\left\{A_i\right\}$ tale che $\bigcup_i A_i = K$.
\`E chiaro come ogni ricoprimento aperto di $C$ $\left\{U_i\right\}_{i \in I}$
tale che $K \subseteq \bigcup_{i \in I} U_i$ debba essere anche un
ricoprimento di $K$, quindi per ipotesi ammette sottoricoprimento finito
$\left\{U_i\right\} : \bigcup_i U_i = K$ la cui intersezione con $C$ è
sicuramente un suo sottoricoprimento aperto finito
$\left\{U_i \cap C\right\} : \bigcup_i U_i \cap C = C$.

Nel caso opposto $K \nsubseteq \bigcup_{i \in I} U_i$ basta considerare
come ricoprimento $\left\{U_i \right\}_{i \in I} \cup C^c$, che è
ancora aperto in quanto unione di aperti ($C^c = K \setminus C$ aperto perché
$C$-chiuso per ipotesi). Questo è un ricoprimento di $K$, quindi come prima
per compattezza sappiamo che ammette sottoricoprimento finito
$\left\{U_i \cup C^c \right\}_i :  \bigcup_i U_i \cup C^c =K$ la cui
restrizione a $C$ è un suo sottoricoprimento aperto finito.
\end{proof}
\end{exercise}

\begin{exercise}
Mostra che il segmento $[0, 1]$ è connesso, usando solo la definizione di
connesso (e nessun altro teorema: di solito questo fatto si mostra subito dopo
la definizione).
\begin{proof}
Per assurdo supponiamo che $[0, 1]$ sia unione disgiunta di due sottoinsiemi
aperti $A, B \subseteq [0, 1] \; : A \cup B = [0, 1]$ con
$A \neq \emptyset$, $B \neq \emptyset$ e $A \cap B = \emptyset$.
Supponiamo che $0 \in A$, poiché $A$ è aperto $\exists \epsilon > 0$
tale che un intorno $U(0)_\epsilon = [0, \epsilon) \subseteq A$.
Consideriamo il $\sup_\epsilon \{U(0)_\epsilon\} = U(0)_\eta$ di questi
intorni; per ipotesi dev'essere $\eta \neq 1$ (altrimenti $[0, 1)$ sarebbe
contenuto in $A$, dunque $A = [0, 1]$ chiuso). Ora, poiché 
$A \cap B = \emptyset$ sono aperti disgiunti, $\eta \not\in B$, ma
$\eta \in A$. Ma allora sempre per apertura di $A$ dovrebbe esistere un
intorno di $\eta$ contenuto in $[0, 1)$ e quindi un secondo raggio 
$\eta' > \eta$ per cui vale ancora $U(0)_\eta' = [0, \eta') \subseteq A$.
Assurdo per definizione di $\eta$ come $\sup$. Da cui concludiamo che
$\eta \in A$, $A = [0, 1]$ e $B = \emptyset$, cioè $[0, 1]$ non è
sconnesso.

\end{proof}
\end{exercise}

\begin{exercise}
Mostra che il sottoinsieme seguente in $\R^2$ è connesso ma non connesso per
archi:
\[
X = \left\{(0, y) | y \in [-1, 1]\right\} \bigcup
\left\{(x, \sin{\nicefrac{1}{x}}) | x > 0\right\}
.\]
\begin{proof}[Svolgimento]

\end{proof}
\end{exercise}

\begin{exercise}
Scrivere le funzioni di transizione di uno dei due atlanti che abbiamo scelto
per $S^n$ e verifica che sono lisce.
\begin{proof}[Svolgimento]

\end{proof}
\end{exercise}

\begin{exercise}
Mostra che la mappa
\[
f: S^n \to \R\PP^n, (x_1, x_2, \ldots, x_{n+1}) \mapsto
[x_1, x_2, \ldots, x_{n+1}]
\]
è liscia.
\begin{proof}[Svolgimento]

\end{proof}
\end{exercise}

Un \emph{diffeomorfismo} è una mappa liscia $f: M \to N$ fra varietà lisce
che ha un'inversa, anch'essa liscia.

\begin{exercise}
Costruisci due atlanti \emph{non} compatibili per la varietà topologica $\R$.
Mostra però che le due varietà lisce risultanti sono comunque diffeomorfe!
\begin{proof}[Svolgimento]

\end{proof}
\end{exercise}

\begin{exercise}
Mostra che $\R\PP^1$ e $S^1$ sono diffeomorfi.
\begin{proof}[Svolgimento]

\end{proof}
\end{exercise}

\section{Esercizi del 16/10/2021}
\begin{exercise}
Costruisci un embedding del toro $n$-dimensionale
\[
S^1 \times S^1 \times \cdots \times S^1 \hookrightarrow \R^{n + 1}
.\] 
per ogni $n \geq 1$.
\begin{proof}[Svolgimento]

\end{proof}
\end{exercise}

Un sottoinsieme $Y$ di uno spazio topologico $X$ è \emph{denso} se interseca
qualsiasi aperto di $X$

\begin{exercise}
Siano $p$, $q$ due numeri reali con $\dfrac{p}{q}$ irrazionale. Mostra che la
mappa
\[
f : \R \to S^1 \times S^1 \\
t \mapsto \left( e^{ipt}, e^{iqt} \right)
\] 
è una immersione iniettiva ma non un embedding: l’immagine è densa in
$S^1 \times S^1$ quindi non può essere una sottovarietà.
\begin{proof}[Svolgimento]
    
\end{proof}
\end{exercise}

\begin{exercise}
Sia $f: \R\PP^2 \to \R^4$ la mappa
\[
f([x, y, z]) = \frac{1}{x^2 + y^2 + z^2} (x^2 - y^2, xy, xz, yz)
\] 
Mostra che $f$ è un embedding.
\begin{proof}[Svolgimento]
    
\end{proof}
\end{exercise}

\begin{exercise}
Sia $S^{2n - 1} \subset \R^{2n}$ una sfera di dimensione dispari. Considera il
campo vettoriale tangente su $S^{2n - 1}$:
\[
X(x_1, x_2, \ldots, x_{2n}) =
\left(x_2, -x_1, x_4, -x_3, \ldots, x_{2n}, -x_{2n - 1}\right)
\] 
Scrivi esplicitamente il flusso di questo campo e determina le sue linee
integrali.
\begin{proof}[Svolgimento]
    
\end{proof}
\end{exercise}

\begin{exercise}
Siano $X$ e $Y$ due campi vettoriali in $\R^n$. Mostra che
\[
[X, Y]^i = X^j \frac{\partial Y^i}{\partial x^j} -
Y^j \frac{\partial X^i}{\partial x^j}
.\] 
\begin{proof}[Svolgimento]
    
\end{proof}
\end{exercise}

\begin{exercise}
Data una matrice quadrata $A$, sia $X_A$ il campo vettoriale su $\R^n$ dato da
$X_A(x) = Ax$. Mostra che
\[
[X_A, X_B] = X_{BA - AB}
.\] 
\begin{proof}[Svolgimento]
    
\end{proof}
\end{exercise}

\begin{exercise}
Dimostra l'identità di Jacobi: dati tre campi vettoriali $X$, $Y$, $Z$ su
una varietà $M$, vale
\[
[[X, Y], Z] + [[Y, Z], X] + [[Z, X], Y] \equiv 0
.\] 
\begin{proof}[Svolgimento]
    
\end{proof}
\end{exercise}

\begin{exercise}
Sia $M$ una varietà, siano $X$, $Y$ campi vettoriali su $M$ e $f$,
$g \in C^\infty (M)$. Mostra che
\[
[fX, gY] = fg[X, Y] - f(Xg)Y - g(Yf)X
.\] 
\begin{proof}[Svolgimento]
    
\end{proof}
\end{exercise}

\end{document}
