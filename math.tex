% abbreviation for (sub_/super^)scripts of \lim, \sum,... in inline math
\newcommand{\ds}{\displaystyle}
\newcommand{\numberset}{\mathbb}
\newcommand{\C}{\numberset{C}}
\newcommand{\R}{\numberset{R}}
\newcommand{\Q}{\numberset{Q}}
\newcommand{\Z}{\numberset{Z}}
\newcommand{\N}{\numberset{N}}
\newcommand{\PP}{\mathbb{P}}
\DeclarePairedDelimiter{\norm}{\lVert}{\rVert}
\DeclarePairedDelimiter{\scalar}{\langle}{\rangle}

% Logarithm with arbitrary base.
% -> log_10
\newcommand{\llog}[1][10]{\log_{#1}}

% Absolute value.
% -> |x|
\newcommand{\abs}[1]{\left| #1 \right|}

% Powers.
% -> x^a
\newcommand{\power}[2][2]{\left( #2 \right)^{#1}}

% Square.
% -> x^2
\newcommand{\sq}[1]{\power[2]{#1}}

% Expansion of the binomial coefficient.
% -> n1!/(n2!(n1 - n2)!)
\newcommand{\binomexpr}[2]{\frac{#1!}{#2!(#1 - #2)!}}

% Expression evaluation at a given point with square brackets.
% -> [x]_{a}
\newcommand{\at}[2]{\left[ #1\right]_{\makebox[-1pt][l]{${\scriptstyle#2}$}}}

% Expression evaluation in an interval.
% -> [x] _{a}^{b}
\newcommand{\eval}[3]{\left.#1%
  \right|_{\makebox[-1pt][l]{${\scriptstyle#2}$}}^{\makebox[-1pt][l]{${\scriptstyle#3}$}}}

% Upright d in math mode (for differentials).
% -> d
\newcommand{\ud}{\mathrm{d}}

% Differential.
% -> dx
\newcommand{\diff}[1][x]{\,\ud{#1}}

% Base command for defining derivatives.
% -> df/dx or d^kf/dx^k
\newcommand{\basederivative}[4][]{%
  \displaystyle%
  \ifx\\#1\\\frac{#4#2}{#4#3}%
  \else%
  \frac{#4^#1#2}{#4#3^#1}%
  \fi%
}

% Total derivative.
% -> df/dx(x) or d^kf/dx^k(x)
\newcommand{\td}[4][]{%
  \basederivative[#1]{#2}{#3}{\ud}%
  \ifx\\#4\\%
  \else%
  \mkern-4mu\left(#4\right)%
  \fi%
}

% Partial derivative.
% -> df/dx(x) or d^kf/dx^k(x)
\newcommand{\pd}[4][]{%
  \basederivative[#1]{#2}{#3}{\partial}%
  \ifx\\#4\\%
  \else%
  \mkern-4mu\left(#4\right)%
  \fi%
}

\newcommand{\prob}[1]{\displaystyle P\left(#1\right)}

\newcommand{\pvalue}{\emph{$p$-value}}

\newcommand{\cond}{\,|\,}

\newcommand{\expect}[1]{\displaystyle E\left[#1\right]}

\newcommand{\mom}[2][]{\displaystyle {\cal M}_{#2}\ifx\\#1\\\else(#1)\fi}

\newcommand{\momalg}[1]{\displaystyle \lambda_{#1}}

\newcommand{\momcen}[1]{\displaystyle \mu_{#1}}

\newcommand{\skewness}{\displaystyle \gamma_1}

\newcommand{\kurtosis}{\displaystyle \gamma_2}

\newcommand{\charf}[1][x]{\phi_{#1}}

\newcommand{\momgenf}[1][x]{M_{#1}}

\newcommand{\fwhm}{{\scriptstyle \textsc{FWHM}}}

\newcommand{\hwhm}{{\scriptstyle \textsc{HWHM}}}

\newcommand{\median}{\mu_{\nicefrac{1}{2}}}

\newcommand{\var}[1]{\ensuremath{\text{Var}\left(#1\right)}}

\newcommand{\cov}[2]{\ensuremath{\text{Cov}\left(#1, #2\right)}}

\newcommand{\corr}[2]{\ensuremath{\text{Corr}\left(#1, #2\right)}}

\newcommand{\like}{\mathcal L}

\newcommand{\likelihood}[2][]{\like\ifx\\#2\\\else(#2\ifx\\#1\\\else;#1\fi)\fi}

\newcommand{\chisq}{\ensuremath{\chi^2}}

\newcommand{\chisquare}[2][]{\chisq\ifx\\#2\\\else(#2\ifx\\#1\\\else;#1\fi)\fi}

\newcommand{\loglikelihood}[2][]{\log\likelihood[#1]{#2}}

\newcommand{\pdf}[3][]{#2(#3\ifx\\#1\\\else;#1\fi)}

\newcommand{\binomialpdf}[2][]{\pdf[#1]{\mathcal B}{#2}}

\newcommand{\multinomialpdf}[2][]{\pdf[#1]{\mathcal M}{#2}}

\newcommand{\poissonpdf}[2][]{\pdf[#1]{\mathcal P}{#2}}

\newcommand{\uniformpdf}[2][]{\pdf[#1]{u}{#2}}

\newcommand{\exponentialpdf}[2][]{\pdf[#1]{\varepsilon}{#2}}

\newcommand{\gausspdf}[2][]{\pdf[#1]{N}{#2}}

\newcommand{\chisquarepdf}[2][]{\pdf[#1]{\wp}{#2}}

\newcommand{\cauchypdf}[2][]{\pdf[#1]{c}{#2}}

\newcommand{\erf}[1]{\ensuremath{\text{erf}\left(#1\right)}}

\newcommand{\dccases}[4][]{#2 \ifx\\#2\\\else=\fi %
  \begin{cases}
    \displaystyle #3 & \text{per variabili discrete}\\
    \displaystyle #4 & \text{per variabili continue}#1
  \end{cases}
}